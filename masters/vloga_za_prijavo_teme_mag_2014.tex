\documentclass[a4paper, 12pt]{article}
\usepackage[slovene]{babel}
\usepackage{lmodern}
\usepackage[T1]{fontenc}
\usepackage[utf8]{inputenc}
\usepackage{url}
\usepackage{enumitem}

\topmargin=0cm
\topskip=0cm
\textheight=25cm
\headheight=0cm
\headsep=0cm
\oddsidemargin=0cm
\evensidemargin=0cm
\textwidth=16cm
\parindent=0cm
\parskip=12pt

\renewcommand{\baselinestretch}{1.2}

\begin{document}

\noindent
% IZPOLNI KANDIDAT
Iztok Jeras\\
Vpisna številka: 60606060\\
Dvorakova ulica 11, 1000 Ljubljana\\
Slovenija


\bigskip

{\bf Komisija za študijske zadeve}\\
Univerza v Ljubljani, Fakulteta za računalništvo in informatiko\\
Večna pot 113, 1000 Ljubljana\\

{\Large\bf
{\centering
    Vloga za prijavo teme magistrskega dela \\%[2mm]
\large Kandidat: Iztok Jeras \\[10mm]}}


Podpisani/-a študent/-ka magistrskega programa na Fakulteti za računalništvo in informatiko, zaprošam Komisijo za študijske zadeve, da odobri temo dela, podrobno opisanega v nadaljnjem predlogu teme magistrskega dela.

Okvirni naslov magistrskega dela:

\hfill\begin{minipage}{\dimexpr\textwidth-2cm}
slovensko: {\bf Predslike 2D celičnih avtomatov}\\
angleško: {\bf Preimages of 2D cellular automata}
\end{minipage}

Za mentorja/mentorico predlagam:

\hfill\begin{minipage}{\dimexpr\textwidth-2cm}
Branko Šter, prof. dr. \\
Fakulteta za računalništvo in informatiko \\
branko.ster@fri.uni-lj.si
\end{minipage}

Za somentorja/somentorico predlagam:

\hfill\begin{minipage}{\dimexpr\textwidth-2cm}
Ime in priimek, naziv: \\
Ustanova: \\
Elektronski naslov: \\
\end{minipage}

% Izpolnite zgolj v primeru pisanja v tujem jeziku:
Komisijo zaprošam, da odobri pisanje magistrskega dela v angleškem jeziku z obrazložitvijo ... .



\bigskip

V Ljubljani, dne …………………………

Podpis mentorja: \hspace{180px} Podpis kandidata/kandidatke:




\clearpage
\section*{PREDLOG TEME MAGISTRSKEGA DELA}

\section{Področje magistrskega dela}

slovensko: diskretni dinamični sistemi, celični avtomati, teoretična študija \\
angleško: discrete dynamic systems, cellular automata, theoretical study


\section{Ključne besede}

slovensko: celični avtomati, predslike, reverzibilnost, trid, quad   \\
angleško: cellular automata, preimages, reversibility, Garden of Eden,  trid, quad


\section{Opis teme magistrskega dela}

\subsection{Uvod in opis problema}

Ker lahko vsak univerzalen sistem modelira vsak drugi univerzalen sistem, predpostavimo,
da lahko s celičnimi avtomati modeliramo vesolje. Pri tem mene najbolj zanimaja pogled s
stališča informacijske teorije in termodinamike. Kakor primera bi lahko navedel kopiranje informacij
(DNK, evolucija /cite{evoloop}), ter model gravitacije kakor entropijske sile (Entropic gravity).

Informacijsko dinamiko celičnega avtomata se najpogosteje opisuje samo kakor reverzibilno ali ireverzibilno,
obstajajo tudi nekaj člankov, ki opazujejo enropijo sistema.
Pogosto je opazovanje dinamike delcev pri Game of Life ali elementarnem pravilu 110.
Ne obstaja pa še splošna teorija dinamike informacij v celičnem avtomatu.

Cilj te naloge je opis algoritma, ki omogoča iskanje predslik (preteklih stanj) sistema, torej evolucijo
ireverzibilnega celičnega avtomata v nasprotno smer od definicije časa.

\subsection{Pregled sorodnih del}

Paulina Léon in Genaro Martínez \cite{PaulinaGenaro2016} poizkušata aplicirati
de Bruijn-ove diagrame na 2D celične avtomate. Točneje opazujeta dva celična avtomata:
'Game of Life' in 'Diffusion rule', s poudarkom na opazovanju stabilnih delcev.

Razni avtorji \cite{Hartman2013} iščejo 'Garden of Eden' vzorce v 'Game of Life'.
Zanimiv je pristop z teorijo končnih avtomatov in regularnih jezikov, ki je v
osnovi namenjen eno dimenzionalnim sistemom. Jean Hardouin-Duparc \cite{} ga je razširil
tako, da je vrstice 2D polja uporabil kakor simbole v regularnem jeziku, zaporedje več
vrstic pa predstavlja besedo. Podoben pristop nameravam uporabiti tudi sam.

Opiral se bom tudi na ideje iz svojih prejšnjih prispevkov \cite{Jeras2007}
\cite{DBLP:conf/iccS/JerasD06} \cite{DBLP:conf/automata/Jeras08},
ki so analizirali problem v eni dimenziji.

\subsection{Predvideni prispevki magistrske naloge}

Doslej sem že razvil napredne algoritme za izračun predslik 1D sistema.
Skozi zgodovino so taki algoritmi napredovali, tako da je padala njihova
procesna zahtevnost in opisna/implementacijska zahtevnost.
\begin{enumerate}[noitemsep,nolistsep]
\item 'brute force' algoritmi \( O(c^n) \)
\item improvizirani algoritmi
\item zasnove matematičnega modela
\item optimalni algoritmi \( O(n \log(n)) \) ali celo \( O(n) \)
\end{enumerate}
Iskanje slik 2D sistema je trenutno nekje med improvizacijo in matematičnim modelom.
Z magistersko nalogo bi rad razvil algoritme, ki se nagibajo k optimalnosti.

\subsection{Metodologija}

Magistersko delo bo obsegalo matematičen model, ki bo za lažje razumevanje tudi grafično predstavljen.
Algoritem bo implementiran kakor računalniški program, ki bo omogočal tudi izris grafične prestavitve
problema. Primerjava s sorodnimi deli bo s stališča procesne zahtevnosi algoritma, in glede na to
katere znane probleme bo algoritem sposoben rešiti. Nekaj takih problemov urejenih glede na zahtevnost je:
\begin{enumerate}[noitemsep,nolistsep]
\item določitev ali obstajajo predslike za dano trenutno stanje sistema
\item štetje predslik
\item naštevanje konfiguracij predslik
\item jezik vseh stanj brez predslik
\item vprašanje reverzibilnosti sistema
\item vprašanje univerzalnosti sistema
\end{enumerate}

\subsection{Literatura in viri}
\label{literatura}

%Navodilo:
Tu navedite vse vire, ki jih citirate v predlogu teme. Citiranje naj bo v skladu z
znanstveno-strokovnimi standardi citiranja, na primer, \cite{Zivkovic2004}.
Seznam naj vsebuje vsaj nekaj del, objavljenih v zadnjih petih letih.
Prednostno naj bodo navedene objave s konferenc, revij, oziroma drugih priznanih virov.

\renewcommand\refname{}
\vspace{-50px}
\bibliographystyle{elsarticle-num}
\bibliography{bibliography}


\bigskip

Ljubljana, \today .

\end{document}
