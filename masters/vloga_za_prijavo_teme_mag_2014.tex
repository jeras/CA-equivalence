\documentclass[a4paper, 12pt]{article}
\usepackage[slovene]{babel}
\usepackage{lmodern}
\usepackage[T1]{fontenc}
\usepackage[utf8]{inputenc}
\usepackage{url}
\usepackage{hyperref}
\usepackage{enumitem}

\topmargin=0cm
\topskip=0cm
\textheight=25cm
\headheight=0cm
\headsep=0cm
\oddsidemargin=0cm
\evensidemargin=0cm
\textwidth=16cm
\parindent=0cm
\parskip=12pt

\renewcommand{\baselinestretch}{1.2}

\begin{document}

\noindent
% IZPOLNI KANDIDAT
Iztok Jeras\\
Vpisna številka: 63030393\\
Dvorakova ulica 11, 1000 Ljubljana\\
Slovenija


\bigskip

{\bf Komisija za študijske zadeve}\\
Univerza v Ljubljani, Fakulteta za računalništvo in informatiko\\
Večna pot 113, 1000 Ljubljana\\

{\Large\bf
{\centering
    Vloga za prijavo teme magistrskega dela \\%[2mm]
\large Kandidat: Iztok Jeras \\[10mm]}}


Podpisani študent magistrskega programa na Fakulteti za računalništvo in informatiko,
zaprošam Komisijo za študijske zadeve, da odobri temo dela, podrobno opisanega v nadaljnjem predlogu teme magistrskega dela.

Okvirni naslov magistrskega dela:

\hfill\begin{minipage}{\dimexpr\textwidth-2cm}
slovensko: {\bf Predslike 2D celičnih avtomatov}\\
angleško: {\bf Preimages of 2D cellular automata}
\end{minipage}

Za mentorja predlagam:

\hfill\begin{minipage}{\dimexpr\textwidth-2cm}
Branko Šter, prof. dr. \\
Fakulteta za računalništvo in informatiko \\
branko.ster@fri.uni-lj.si
\end{minipage}

Za somentorja predlagam (še čakamo na uradno potrditev):

\hfill\begin{minipage}{\dimexpr\textwidth-2cm}
Genaro Juárez Martínez, prof. dr. \\
University of the West of England (UWE) \\
genaro.martinez@uwe.ac.uk \\
\end{minipage}

% Izpolnite zgolj v primeru pisanja v tujem jeziku:
Komisijo zaprošam, da odobri pisanje magistrskega dela v angleškem jeziku, da bi ga lahko neprevedenega bral somentor iz tuje univerze.



\bigskip

V Ljubljani, dne …………………………

Podpis mentorja: \hspace{180px} Podpis kandidata:




\clearpage
\section*{PREDLOG TEME MAGISTRSKEGA DELA}

\section{Področje magistrskega dela}

slovensko: diskretni dinamični sistemi, celični avtomati, teoretična študija \\
angleško: discrete dynamic systems, cellular automata, theoretical study


\section{Ključne besede}

slovensko: celični avtomati, predslike, reverzibilnost, trid, quad   \\
angleško: cellular automata, preimages, reversibility, Garden of Eden,  trid, quad


\section{Opis teme magistrskega dela}

\subsection{Uvod in opis problema}

Ker lahko vsak univerzalen sistem modelira vsak drugi univerzalen sistem, predpostavimo,
da lahko s celičnimi avtomati modeliramo vesolje. Pri tem mene najbolj zanima pogled s
stališča informacijske teorije in termodinamike. Kakor primera bi lahko navedel kopiranje informacij
(DNK, evolucija \cite{Salzberg2004}), ter model gravitacije kakor entropijske sile (Entropic gravity).

Informacijsko dinamiko celičnega avtomata se najpogosteje opisuje samo kakor reverzibilno ali ireverzibilno,
obstaja tudi nekaj člankov, ki opazujejo entropijo sistema.
Pogosto je tudi opazovanje dinamike delcev pri Game of Life ali elementarnem pravilu 110.
Ne obstaja pa še splošna teorija dinamike informacij v celičnem avtomatu.

V svojem članku \cite{JerasDobnikar2007} sem grafično upodobil predslike trenutnega stanja za 1D problem.
Iz upodobitve je videti, da se ponekod izgbubi več informacije kakor drugod,
kar kaže na možnost izpeljave kvalitativne in kvantitativne teorije dinamike informacij;
žal se ta možnost še ni udejanila.

Cilj te naloge je opis algoritma, ki omogoča iskanje predslik (preteklih stanj) sistema, torej evolucijo
ireverzibilnega celičnega avtomata v nasprotno smer od definicije časa.
Upodobitev predslik 2D sistema bo razširitev upodobitve za 1D sistem
in bo ponujala podoben vpogled v dinamiko informacij 2D sistema.

\subsection{Pregled sorodnih del}

Paulina Léon in Genaro Martínez \cite{PaulinaGenaro2016} poizkušata aplicirati
De Bruijn-ove diagrame na 2D celične avtomate. Točneje, opazujeta dva celična avtomata:
'Game of Life' in 'Diffusion rule', s poudarkom na opazovanju stabilnih delcev.
De Bruijn-ovi grafi so tudi osnova mojih raziskav, so pa drugače grafično upodobljeni,
tako da se jih lahko poveže v opis celotnega celičnega sistema, in niso omejeni na
opis predstanj ene same celice.

Razni avtorji \cite{Hartman2013} iščejo vzorce tipa 'Garden of Eden' v celičnem avtomatu 'Game of Life'.
Zanimiv je pristop s teorijo končnih avtomatov in regularnih jezikov, ki je v
osnovi namenjen eno dimenzionalnim sistemom. Jean Hardouin-Duparc ga je razširil
tako, da je celice iz vrstice 2D polja združil v simbole regularnega jezika, zaporedje več
vrstic pa predstavlja besedo. Originalni članek je v francoščini, zato še iščem članek,
kjer bi pristop opisal v angleščini. Podoben pristop s končnimi avtomati nameravam uporabiti tudi sam.

Opiral se bom tudi na ideje iz svojih prejšnjih prispevkov \cite{JerasDobnikar2007}
\cite{DBLP:conf/iccS/JerasD06} \cite{DBLP:conf/automata/Jeras08},
ki so analizirali problem v eni dimenziji.

\subsection{Predvideni prispevki magistrske naloge}

Doslej sem že razvil napredne algoritme za izračun predslik 1D sistema.
Skozi zgodovino so taki algoritmi napredovali, tako da je padala njihova
procesna zahtevnost in opisna/implementacijska zahtevnost.
\begin{enumerate}[noitemsep,nolistsep]
\item 'brute force' algoritmi \( O(c^n) \)
\item improvizirani algoritmi
\item zasnove matematičnega modela
\item optimalni algoritmi \( O(n \log(n)) \) ali celo \( O(n) \)
\end{enumerate}
Iskanje slik 2D sistema je trenutno nekje med improvizacijo in matematičnim modelom.
Z magistrsko nalogo bi rad razvil algoritme, ki se nagibajo k optimalnosti.

\subsection{Metodologija}

Magistrsko delo bo obsegalo matematičen model, ki bo predvidoma temeljil na matričnih operacijah,
kjer matrike predstavljajo grafe in končne avtomate.
Za lažje razumevanje bo problem tudi grafično predstavljen.
Algoritem bo implementiran kot računalniški program, ki bo omogočal tudi izris grafične predstavitve
problema.

Primerjava s sorodnimi deli bo s stališča procesne zahtevnosi algoritma in glede na to,
katere znane probleme bo algoritem sposoben rešiti. Nekaj takih problemov, urejenih glede na zahtevnost, je:
\begin{enumerate}[noitemsep,nolistsep]
\item določitev, ali obstajajo predslike za dano trenutno stanje sistema
\item štetje predslik
\item naštevanje konfiguracij predslik
\item jezik vseh stanj brez predslik
\item vprašanje reverzibilnosti sistema
\end{enumerate}

Rešitev problema določitve obstoja predslik si že predstavljam. 
Predvidevam, da bom uspel rešiti še problem preštevanja predslik, in ker je to manjši korak, tudi njihovo naštevanje.

Preostalih problemov se tokrat ne bom loteval.
Problem jezika stanj brez predslik bi potreboval model 2D formalnega jezika, ki še ne obstaja.
Poleg tega je povezan s problemom reverzibilnosti, ki je na spološno dokazano nerešljiv, podobno kakor problem tlakovanja (Wang tile).

\subsection{Literatura in viri}
\label{literatura}

\renewcommand\refname{}
\vspace{-50px}
\bibliographystyle{elsarticle-num}
\bibliography{vloga_za_prijavo_teme_mag_2014}


\bigskip

Ljubljana, \today .

\end{document}
